%pdflatex-../thesis.tex
% vim:spell spelllang=en_us

Virtualization is, in my opinion, the most important technology in data centers because it caused significant progress in the field. It is not technology itself, so it should rather be called an architecture than a technology.

Definition of virtualization as stated in \cite{virtualization-in-education} says that \Uv{virtualization is a technique for hiding the physical characteristics of computing resources from the way how systems, applications or end users interact with those resources. The concept of virtualization is very broad and covers devices, servers, operating systems, applications and even networks.} This definition describes the virtualization and can be applied to any type of virtualization.

Computer virtualization it it the oldest and the most common type of virtualization.
It started in 1960s with mainframes as an attempt to employ resource sharing and the idea is still alive in current time. 
Virtual computer is logical representation of physical computer in software. \cite{virtualization-in-education} Virtual computers are usually called virtual machines (\Ac{VM}s) and physical machines hosting \Ac{VM}s are called hosts. 
%The Rhigorous term for physical hosting machine is host and hypervisor is software performing the virtualization, but word hypervisor is widely used in technical text for machines as well. 
The physical computer which is used only to run virtual machines is sometimes called a hypervisor. However this terminology is not right, because hypervisor is a software for running the virtual machines.

It is possible and very advantageous to host many virtual machines on single physical computer because it brings technical and economical benefits. Decoupling a computer and it's software from hardware is important advantage because it brings additional level of abstraction and gives ability to shift virtual machines between hypervisors. Economical benefit is quite obvious, since it is not necessary to buy single physical server for every service and electricity saving is also appreciable.

Another important type of virtualization is virtualization of networks. It is typically used together with computer virtualization, since it can decouple network devices from network itself. Physical machines are not as flexible as \Ac{VM}s are, so plugging them into virtual network is not as beneficial as \Ac{VM}s because there are still physical network cables that can be hardly virtualized. Software Defined Networking (\Ac{SDN}) is currently popular topic and it is having a potential to provide virtualization into physical network infrastructure, thus it may be good idea to integrate physical machines into virtual network as well.

Storage virtualization should also be taken into account because it provides abstraction of the storage. Typical unvirtualized storage uses physical device for storing data and metadata, but this approach is not flexible enough since it is usually limited to just one physical machine or group of machines connected to shared storage. It is necessary to find an agile and scalable storage virtualization technique.

Service virtualization, memory virtualization, I/O virtualization or database virtualization are another types of virtualization. It is not possible to enumerate all types of virtualization because it is fairly possible to virtualize almost everything so new type of virtualization will probably arise.

Term virtualization is will be used in meaning of computer virtualization, other types of virtualization will always be denoted.
