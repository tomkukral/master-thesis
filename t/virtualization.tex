%pdflatex-../thesis.tex
% vim:spell spelllang=en_us

Virtualization is, in my opinion, the most important technology in data centers, because it caused significant progress in this field. It is not technology itself, so it should rather be called model than technology.

Definition of virtualization as stated in \cite{virtualization-in-education} says that \Uv{virtualization is a technique for hiding the physical characteristics of computing resources from the way in which other systems, applications or end users interact with those resources. The concept of virtualization is very broad and can be applied to devices, servers, operating systems, applications and even networks.} This definition gives description of the virtualization and can be applied to any type of virtualization.

The most common approach is virtualization of computers, because it is the oldest one and most widely used there days. It started in 1960s with mainframes as an attempt to employ resource sharing and this idea is still alive in current time. Virtual computer is logical representation of computer in software. \cite{virtualization-in-education} Virtual computers are usually called virtual machines (\Ac{VM}) and physical machine hosting \Ac{VM}s is called hypervisor. It is possible and very advantageous to host many virtual machines on single physical computer, because it brings technical and economical benefits. Decoupling computer and it's software from hardware is important advantage, because it brings additional level of abstraction and allows you to shift virtual machines between hypervisors. Economical benefit is quite obvious, since it is not necessary to buy single physical server for single service and electricity saving are also appreciable.

Another type of virtualization is virtualization of networks. It is usually used together with computer virtualization, since it gives an occasion to separate network devices from network itself. Physical machines are not as flexible as \Ac{VM}s are, so plugging them into virtual network is not as beneficial as \Ac{VM}s, because there are still physical network cables, that can be hardly virtualized. There is a hot topic called Software Defined Networking (\Ac{SDN}) having potential to provide virtualization info physical network infrastructure, thus it may be good idea to integrate physical machines into virtual network as well.

Storage virtualization should also be taken into account, because it provides abstraction of the storage. Typical unvirtualized storage uses some physical device for storing data and metadata, but this approach is not enough flexible since it is usually limited to just one physical machine or group of machines connected to shared storage. It is necessary to find any method of storage virtualization, which would be able to connect any storage to any physical of virtual computer.

Service virtualization, memory virtualization or database virtualization are another types of virtualization. It is not necessary to mention all the types of virtualization since it is possible to virtualize almost everything and emerging of new types is quite probable. 

