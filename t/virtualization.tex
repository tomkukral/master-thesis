%pdflatex-../thesis.tex
% vim:spell spelllang=en_us

Virtualization is, in my opinion, the most important technology in data centers which caused significant progress in this field. It is not technology itself, so it should rather be called model than technology.

Definition of virtualization as stated in \cite{virtualization-in-education} says that \Uv{virtualization is a technique for hiding the physical characteristics of computing resources from the way in which other systems, applications or end users interact with those resources. The concept of virtualization is very broad and can be applied to devices, servers, operating systems, applications and even networks.} I brings basic suggestions, but also mention broad topic and blurred borders of virtualization. 

The most common approach is virtualization of computers, because it is the oldest one and most widely used there days. It started in 1960s with mainframes as an attempt to employ resource sharing and the idea is still alive in current time. Virtual computer is logical representation of computer in software. \cite{virtualization-in-education} Virtual computers are usually called virtual machines (\Ac{VM}) and physical machine hosting \Ac{VM}s is called hypervisor. It is possible and very advantageous to run many virtual machines on one physical computer, because it brings technical and economical benefits. Decoupling computer and it's software from hardware is important advantage, because it brings additional level of abstraction and allows to shift virtual machines between hypervisors. Economical benefit is quite obvious, since it is not necessary to buy physical server for every service and electricity saving are also appreciable.



Virtualization of networks

Virtualization of services
