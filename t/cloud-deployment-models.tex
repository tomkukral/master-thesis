%pdflatex-../thesis.tex
% vim:spell spelllang=en_us

There are three scenarios possible for deploying cloud solutions. Models differs by ownership and subject responsible by administration of the system. Right solution depends on expected load, available budget as well as on expected classification of data. Public model and private model are mutually contradictory and last model called hybrid is combination of first two mentioned.

\subsubsection{Private}
Private model defines cloud environment build exclusively for single subject. Typical scenario is to build private cloud in datacenter owner by the subject, but it is not strictly required. There is common misunderstanding of term private, because it means private usage of cloud resources and not private ownership of cloud infrastructure. Private cloud may be leased from third-party provider and it also can be running on third-party hardware.

Running private cloud gives an advantage in elimination of any inter-tenant isolation problems and it is possible to adapt configuration to fit owner needs. There is a law, which forces sensitive data to be stored internally and with limited access, so it is necessary to build private or hybrid cloud for this kind of usage. It is not clear how to handle clouds and especially storages with law, because it this topic is wide and it is not possible to define rigid rules. There is currently running case with \Ac{US} judge ordering Microsort to provide data stored in Ireland. \cite{wp-microsoft-ireland}

Drawback of private cloud is higher initial cost and probably also higher operational costs, but it depends on expected usage.

\subsubsection{Public}
Public cloud deployment model is based on resource sharing between the tenants. There is usually one subject called cloud provider and many customers (tenants). These tenants buy resources and rights to use them. Resources are usually charged according to it's usage.

Billing resource per use is called pay-per-use and it is interesting method of balancing or shifting costs between initial and operational. There are usually plans with various \Ac{CPU}, memory and storage option and final cost depends on real usage of resource. It is most significant difference compared to public cloud, because 

Infrastructure is not dedicated and it is shared between tenants. Resources are shared, but must be strictly isolated, because it is unacceptable to allow any interference between tenants, unless they make an explicit request to allow it.

\subsubsection{Hybrid}




%\begin{table}[htb]
%\begin{center}
%	\caption{Comparison of deployment models}
%	\label{tab:deployment-models}
%	\begin{tabularx}{\textwidth}{|l|X|c|X|}
%	\hline
%	\Th{Type} & \Th{method}  & \Th{guest modif.} & \Th{usage} \\
%	\hline
%	Paravirtualization & hypercalls by guest kernel & yes & same workloads and same \Ac{OS}\\
%	\hline
%	Full & translation of instructions & no & when full abstraction is needed \\
%	\hline
%	Hardware assisted & translation with help of hardware & no & same as full, but with compatible \Ac{CPU}\\
%	\hline
%	Hybrid & translations and driver changes & driver only & when possible to install additional drives\\
%	\hline
%	\end{tabularx}
%\end{center}
%\end{table}

