%pdflatex-../thesis.tex
% vim:spell spelllang=en_us

It is possible to find many services called \Uv{cloud based} and it is important to agree on accurate definition of these services. It is quite clear that cloud based service will use principle of cloud computing. Definition of cloud computing by \Ac{NIST} says, that \Uv{Cloud computing is a model for enabling ubiquitous, convenient, on-demand network access to a shared pool of configurable computing resources (e.g., networks, servers, storage, applications and services) than can be rapidly provisioned and released with minimal management effort or service provide interaction.} \cite{cloud-definition}. This definition clarifies what cloud computing is, but says nothing about parameters and used technologies.

I think that it would be more convenient to start definition from lower levels, which provides elementary parts, and get to the cloud service afterwards. This definition gives different look at cloud computing than \Ac{NIST}s, but it uses same conditions and therefore results are basically the same. It focuses on currently used principles and provides more technical description of cloud services.

Cloud computing services are nowadays heavily dependent on virtualization because it allows to replace physical machines with virtual machines (\Ac{VM}s) or containers and brings a lot more flexibility than physical machine can ever provide.

Basic element of cloud computing system is virtual machine. Physical machine can also be part of the cloud system, but it is not able to deliver required rapid provisioning and it is not possible to deploy physical machine without service provider interaction. Virtual machine uses additional resources. These resources can be for example networking, which is used for interconnection between \Ac{VM}s as well as for reaching customers or storage used for system internal or customer data. It is important to employ some configuration management and orchestration because it is able to deliver rapid provisioning of virtual machines and minimizes effort required for administration.

Virtual machines together provide the service, which is exposed to users via any kind of network. It doesn't matter whether customers access the service directly or via a proxy, but hiding \Ac{VM}s brings additional flexibility for migration and scalability. 

Difference between cloud computing and bare virtualization is intelligence included in cloud because it is controlled automatically according to events or monitoring observed at cloud system. It is common to provide configuration interface, which allows customers to tune service parameters.
Bare virtualization does not offer any intelligence, even if it is equipped with shiny user interfaces and opportunity to scale virtual machines up or down because changes are performed manually.

Cloud computing is very popular topic, so it is often used just for marketing purposes and thus it is recommended to perform service analysis and do no trust every buzzword used in marketing materials.
