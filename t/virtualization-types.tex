%pdflatex-../thesis.tex
% vim:spell spelllang=en_us

There are three different virtualization types and they vary by method used t
add virtualization layer between guests and physical hardware. It it not possible to easily choose better or worse virtualization types, because it depends on intended usage, character of computing tasks and required operating system.
 
Architectures of computers, especially x86, are designed to run on physical devices, thus is in not easy to virtualize them. Access to hardware is controlled by priority levels called rings. Lowest priority is used by userspace applications and highest priority (ring 0) is reserved for operating system. It is necessary to insert virtualization layer between operating system and hardware, but there is not any ring with higher priority than operating system uses. This problem needs to be solved and it is not only one challenge. There are sensitive instructions incompatible with virtualization, because they use different semantics when they are not run in ring 0, as mentioned in \cite{vmware-para}.

\subsection{Paravirtualization}
Paravirtualization is type of virtualization with necessity of modifications in guest kernel. Modifications of kernel are necessary, because operating system uses non-virtualizable instructions that are trying to gain direct access to the hardware. These instruction need to be replaced with hypercalls that communicate directly with virtualization layer of hypervisor. \cite{vmware-para} It is obvious, that guest operating system knows it is running virtualized. 

Biggest advantage of paravirtualization is lower overhead compared to other types, because it is not necessary to translate instructions before running. However this advantages becomes less significant during time since there are already available processors optimized to run hardware assisted virtualization with less overhead. Main drawback of this type of virtualization is need for modifications done at an operating system, which is not always possible or allowed. Running modified \Ac{OS} also brings additional administration and thus additional cost.

It is possible to take a different look at paravirtualization and do not try to create entire virtual machine, but use operating system-level virtualization, where kernel allows to run multiple userspaces. These userspaces are called containers and therefore this approach is sometimes called container virtualization. It does not provide entire isolated virtual machine, but allows to run software packed in container. It is advantageous because there is almost none overhead in running software from container while maintaining sufficient level of container isolation. Container virtualization is applicable for situation, where whole virtual machine is not needed and then brings huge performance improvements since operating system layer is shared. Some says, that containers are going to bring next revolution info virtualization. For example Dustin Kirkland, Cloud Solutions Product Manager at Canonical wrote: \Uv{Linux containers, repositories of popular base images, snapshots using modern copy-on-write filesystem features.  Brilliant, yet so simple.  Docker.io for the win!} \cite{ubuntu-docker}. I think, that container virtualization may brings compelling advantages and I also like using it, but it is not suitable for every situation. It is still technically kind of paravirtualization and thus it is limited to provide only additional layer on host's operating system.

\subsection{Full virtualization}
Virtualization type capable or running unmodified operating system is called full virtualization. It utilizes runtime translation, which captures non-virtualizable commands and emulates them using hypervisor virtualization layer. Virtualizable instructions are executed directly on the hardware. Modification of \Uv{problematic} calls is carried by the hypervisor and it is the main difference compared with paravirtualization. 

Most important benefit of full virtualization is it's ability to run guest operating system without any changes, so guest \Ac{OS} is not aware of being virtualized. This makes guest operating system fully abstracted from underlaying hardware, it is possible to multiple different operating system on single host and provides simple migration from physical to virtual machine. Drawback of this type is overhead caused by catching and translating non-virtualizable calls. 

\subsection{Hardware assisted virtualization}
Full virtualization has significant overhead caused by binary translation, so \Ac{CPU} vendors introduced technologies capable of inserting virtualization layer between ring 0 and physical hardware. It speeds-up trap of privileged and sensitive calls to the hypervisor and it is not necessary to perform binary translation of to modificate kernel of guest operating system. 

Benefit of this type is quite obvious, because it lowers virtualization overhead and thus provides better performance compared with full virtualization together with elimination of need for guest kernel modifications compared with paravirtualization. It is necessary to have a support in host's \Ac{CPU} is primary drawback of this type, but there is support in almost every processor in current marker.

Running unmodified guest operating system leaves all necessary translations of instructions on hypervisor layer, so I would be good to to introduce small changes to guest's operating system, which will reduce work left for the hypervisor but also do not need any significant changes in guest's kernel. This approach is called hybrid virtualization and it is subset of hardware assisted virtualization. Installation of additional drivers is required, but it is not necessary to apply any changes on whole kernel. These drivers are aware of virtualization and use virtualization layer directly without any translations made by the hypervisor. This method increases driver's \Ac{IOPS} and therefore it is usually used for virtualized network cards and storages. Driver able to deliver hybrid virtualization is \Name{virtio} for \Ac{KVM}, Xen call it \Name{paravirtualized device drivers} and VMWare  \Name{Guest Tools}.

\subsection{Summary on types of virtualization}
There were presented some virtualization general virtualization types and their pros and cons. There is not any universal virtualization type suitable for all use cases, thus is is always possible to decide on planned usage. It also depends whether it is required to run different kernel on single physical host or it is sufficient to share one kernel for all containers. Differences are compared in table \ref{tab:virtualization-types}.

We can divide types into two groups:
\begin{itemize}
	\item One group provides guests with full virtual machine, every \Ac{VM} uses it's own isolated kernel and \Ac{VM}s are fulll or almost fully decoupled from hardware. Full, hardware assisted and hybrid virtualization belongs to this group.
	\item Members of second group are containers and paravirtualization. This group is specific by lightweight containers and host kernel shared by all running containers.
\end{itemize}

Virtualization is massively used even by czech \Ac{IT} companies. First group is used for example by \Name{Wedos} for their virtual server hosting and related services. Second group is uses by \Name{Seznam.cz} and they use \Ac{LXC} for web serves as well as for Hadoop cluster.



\begin{table}[htb]
\begin{center}
	\caption{Comparison of virtualization types}
	\label{tab:virtualization-types}
	\begin{tabularx}{\textwidth}{|l|X|c|X|}
	\hline
	\Th{Type} & \Th{method}  & \Th{guest modif.} & \Th{usage} \\
	\hline
	Paravirtualization & hypercalls by guest kernel & yes & same workloads and same \Ac{OS}\\
	\hline
	Full & translation of instructions & no & when full abstraction is needed \\
	\hline
	Hardware assisted & translation with help of hardware & no & same as full, but with compatible \Ac{CPU}\\
	\hline
	Hybrid & translations and driver changes & driver only & when possible to install additional drives\\
	\hline
	\end{tabularx}
\end{center}
\end{table}

