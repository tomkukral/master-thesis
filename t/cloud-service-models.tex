%pdflatex-../thesis.tex
% vim:spell spelllang=en_us

% introduction to models
Purpose of cloud computing is to deliver the service and provide customers with tools able to manage this service. Service models differs by level of control provided to customers and thus with areas of responsibility. I am going to call border between responsibility of customer and responsibility of provider as responsibility border. 
Service models set responsibility border at different levels. Some of them leaves almost all control of service and responsibility at provider side and other supplies customer with more control. It is necessary to select right service model according to expected service usage and required control level.

\subsubsection{Infrastructure as a Service}
\Ac{IaaS} is model with the most configuration tasks left at customer side. Customer is responsible for virtual machines and it's services, so it gives much more flexibility than other models and it is well-suited for services with extraordinary requirements. 

Customer manages virtual machines as well as running services, so provider is responsible only for virtualization and underneath layers. It is even possible to run custom operating system, but provider usually offers prepared images with different operating systems. Prepared \Ac{OS} images are usually tested and modified to run well in cloud environment. There can be installed hybrid virtualization drivers, kernel tweaked to run virtualized and it is also good idea to remove useless drivers and software. 

This model is good choice if special configuration is needed, but it is more difficult to deploy services because some expertise is required. \Ac{IaaS} can be used if customer require additional level of security, because virtual machine can use crypted volume and make data unreadable for provider. Unfortunately it is still possible for provider to acquire confidential from other sources, for example from memory, but it is much complicated to perform this.

Typical example of \Ac{IaaS} is Amazon Web Services and Active24's \Uv{Virtuální Privátní servery}.

\subsubsection{Platform as a Service}
\Ac{PaaS} is on level higher than \Ac{IaaS}. Service provider supply infrastructure and services, but running software is controlled by customer. This model is good for running custom services on provider's platform.

\subsubsection{Software as a Service}
\Ac{SaaS} is on 

