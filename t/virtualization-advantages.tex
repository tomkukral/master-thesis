%pdflatex-../thesis.tex
% vim:spell spelllang=en_us

% decoupling
Most important advantages is decoupling software from physical hardware, at least in my opinion. It is possible to migrate virtual machines with running services between physical hosts without significant impact on service behavior. This brings amazing opportunity to adapt service environment on demand and scale the service.

It is possible to perform any hardware and software upgrades, because all running services may be temporarily migrated to other physical host. Virtual machines are much more easier to deploy than physical ones. It takes only a few seconds to create and run \Ac{VM} compared to at least hours to deploy physical machine. Deploy of virtual machine do not have to be performed by persons, because it is possible to employ an orchestration and scale up the service (add virtual machines) automatically.
Reset of virtual machine is actually just software instruction in hypervisor, so it may by done remotely with ease.

Geographical backups or failover is much more easier to accomplish with virtualization approach. You can rent virtual machine from provider in foreign country and start your services in a few moments. It is huge simplification compared with running physical machine at foreign data center.

% utilization
Virtualization brings also some economical and environmental advantages. Economical advantages are quite obvious, because it is no longer necessary to buy physical servers. Non-virtualizational approach requires one physical machine for every running server, but it is not longer necessary with virtualization. It is possible to run many virtual servers or containers on single physical machine. It is also possible to move even to the higher level of \Ac{CAPEX} cutting and rent virtual machine from provider and absolutely eliminate need for running any server machine. Renting virtual server increases \Ac{OPEX}, but they are more flexible and easier to control.
Electrical consumption should also be taken into account, because single physical machine, even under higher load, will definitely consume less power compared with two or more similar machines. 

I asked Petr Hodač, technical manager at SiliconHill and he stated, that they managed to reduce electricity consumption of whole server room by 19\% iter alia due to deployment of virtualization. It produced also additional saving, because they need less \Ac{UPS} batteries and less cooling capacity, but savings on cooling are not included in mentioned savings.

% disadvantages
However there are some drawback too. Failure of physical host causes failure of all virtual machines or containers running on this host. It is kind of single point of failure, but we can fight it with duplication of service nodes between different hosts, datacenters, providers or continents.
Another disadvantage is hidden in additional virtualization level, since it is necessary to take care of hypervisors. I is not a real disadvantage, since traditional non-virtualized approach needs to take a care of many virtual machines.

% final note
Deployment of virtualization should always be well planned, because it can bring many amazing advantages, but it is also able to cause a disaster in case of poor system design or lame administration.
