%pdflatex-../thesis.tex
% vim:spell spelllang=en_us

The goal of practical part is to develop system capable to measure virtual machine availability during migration. 

% virtualization
Virtualization is enabling technology for \Ac{VM} migration since it decouples virtual machine from physical server. However there are some dependencies which must be transfered together with virtual machine such as virtual network and disk image. Transfer of disk image can be solved by shared storage or using cold migration, but it is not very clear how will migration affect networking communication.

% network
Network connected to virtual machine is called virtual network but there need to be physical infrastructure capable to transfer signal between virtual server and other side of communication (usually customer). It is possible to migrate virtual machine to any server almost without any limitations but there may be serious problem with virtual network. 

Virtual machine availability can be measured on many layers. Application layer can be tested by simulating customer requests. There is a program called ApacheBench for example and it is capable to send many parallel request to \Ac{HTTP} servers and measure statistics like time for request, transfered bytes etc. There are other similar benchmarks as dkftpbench or SPECweb99 but they are always limited to one type of service or protocol. I think that this limitation is critical for virtual machine migrations testing and it is necessary to use more general approach.

General measurement procedure should be service and platform independent so only possible solution is to perform measurement directly on network layer. Statistics acquired by network measurement may be approximately converted into higher layers and estimation of service availability can be done. Network measurement can give some information about network environment during migration as well. Packet loss is the most relevant sign for virtual machine availability during migration but there are other parameters which should be taken into account. For example packet delay can significantly affect service quality, especially for storage.

% migration time
Testing of virtual machine migration is extensive topic because there are additional parameters specific for migration of virtual machines. Total migration time may be used to decide whether it is convenient to use live or cold migration. Total migration time is crucial indicator used in emergency migration cases which may include unplanned outage or a natural disaster. Another special parameter is migration success rate since migration request can be unsuccessful in some cases.

I think that it is important to introduce application capable to evaluate virtual machine migrations so I have developed application called Themis\footnote{Themis is Greek Titaness usually depicted holding scales and it is a reason why application is called Themis.}. It is modular framework which can be instructed to provide migrations in defined way, collect performance data and export them to user.
