%pdflatex-this
% vim:spell spelllang=en_us

\documentclass[12pt,oneside,a4paper]{report} %pdf
\usepackage[english]{babel}
\usepackage[top=25mm, bottom=15mm, includefoot, left=30mm, right=30mm]{geometry}
\usepackage[utf8]{inputenc}
\usepackage{graphicx}
\usepackage[unicode=true]{hyperref}
\usepackage{indentfirst}
\usepackage{import}
\usepackage{import}
\usepackage{moreverb}
\usepackage{multirow}
\usepackage{xcolor}
\usepackage{tabularx}
\usepackage{pdfpages}
\usepackage{ifthen}
\usepackage{cite}

\usepackage[toc,acronym,nonumberlist,style=long,sort=standard]{glossaries}
\setlength{\glsdescwidth}{0.7\linewidth}
\renewcommand*{\glsgroupskip}{}
\makeglossaries

% image numbering
\renewcommand{\thefigure}{\thesection.\arabic{figure}}
\renewcommand{\thetable}{\thesection.\arabic{table}}

\newcommand{\Uv}[1]{"#1"}
\newcommand{\Th}[1]{\textbf{#1}}

\newcommand{\Name}[1]{\textit{#1}}


\newcommand{\Acronym}[2]{\newacronym{#1}{#1}{#2}}
\newcommand{\Ac}[1]{\acrshort{#1}}

\def \BookType {Master thesis}
\def \Bookname {Measurements of virtual machines migrations}
\def \Keywords {Data center, overlay, underlay, OpenNebula, migration}
\def \Authors {Bc. Tomáš Kukrál}
\def \Supervisor {doc. Ing. Leoš Boháč, Ph.D.}
\def \Date {October 2015}
\def \Datefull {2014-11-11} % ISO8601
\def \Place {Prague}
\def \School {Czech Technical University in Prague}
\def \Faculty {Faculty of Electrical Engineering}
\def \Department {Department of Telecommunication Engineering}
\def \Programme {Communication Networks} 

\makeatletter
\def\@makechapterhead#1{%
  \vspace*{20\p@}%
  {\parindent \z@ \raggedright \normalfont
    %\ifnum \c@secnumdepth >\m@ne
    %    \huge\bfseries \@chapapp\space \thechapter
    %    \par\nobreak
    %    \vskip 20\p@
    %\fi
    \interlinepenalty\@M
    \Huge \bfseries #1\par\nobreak
    \vskip 40\p@
  }}

\makeatother

\def \Titulkadesky {
    \begin{center}
    {\LARGE \School}\\[2ex]

%    \begin{figure}[h!]
%    \begin{center}
%      {\includegraphics[width=35mm]{grafika/cvut-logo-bw-600.png}}
%    \end{center}
%    \end{figure}
    \vfill

%    {\textbf {\Huge \Bookname \\[4ex]}
        {\LARGE \bf \BookType}
%	    }

  \vspace{60mm}

    {\large \bf \Date \hfill \Authors}\\[6ex]

    \newpage
    \end{center}
    }

\def \Titulka {
    \begin{center}
    {\Huge \School}\\[2ex]
    {\Large \Faculty}\\[2ex]
    {\Large \Department}\\[2ex]

    \begin{figure}[h]
    \begin{center}
      {\includegraphics[width=4cm]{images/cvut-logo-bw-600.png}}
    \end{center}
    \end{figure}
    \vfill

    {\textbf {\Huge \Bookname \\[4ex]}
        {\LARGE \bf \BookType}
	    }

    \vfill

    {\large {\bf Author:} \hfill \Authors}\\[3ex]
    {\large {\bf Supervisor:} \hfill \Supervisor}\\[3ex]

    \newpage
    \end{center}
    }



\hypersetup{
	pdfauthor={\Authors},
	pdftitle={\Bookname},
	pdfsubject={\BookType},
	pdfkeywords={\Keywords},
	colorlinks=true,
%	hidelinks=true,
}

% environment pro abstract, poděkování a prohlášení
\newenvironment{spodnitext}[1]{
	\cleardoublepage
	\null
	\vspace{17cm}
	\section*{#1}
	}{
	\vspace{10mm}
	}


\begin{document}
% acronyms
\subimport{./}{acronyms}


\catcode`\-=12

%\begin{titlepage}
%	\Titulkadesky
%\end{titlepage}

% vypnout watermark

% prázdná stránka
%\newpage
%\null
%\clearpage

\begin{titlepage}
%	\Titulka
\end{titlepage}

\pagenumbering{Roman}

%prohlášení
\begin{spodnitext}{Proclamation}
I declare that this thesis was made by myself with assistance of my supervisor. All parts taken over word by word from literature or other publications are referenced and identified. I approve publishing this thesis or any part of it with referencing author of original text.

\vspace{10mm}
In~\Place~at~\Datefull \hfill \hbox to 65mm{\tiny\dotfill}
\end{spodnitext}

%zadání
%\includepdf[pages=-]{grafika/zadani.pdf}

% abstrakt
\begin{spodnitext}{Abstract}
	This master thesis deals with testing data networks in cloud environment. Techniques of inter and intra-cloud networks are described in theoretical part as well as virtual machine migrations.
	Practical part brings methodology and framework for testing virtual machine migration. Measurements are performed at OpenNebula cloud environment with {KVM} virtual machines.
\end{spodnitext}


%obsah
\tableofcontents
\cleardoublepage


\pagestyle{plain}
\pagenumbering{arabic}

\chapter{Introduction}
%\subimport{./}{introduction}

\chapter{Theoretical part}
	\section{Virtualization}
		\subimport{./t/}{t/virtualization}

		\subsection{Types of virtualization}
			\subimport{./t/}{t/virtualization-types}

		\subsection{Advantages of virtualization}
			\subimport{./t/}{t/virtualization-advantages}
			\label{subsec:advantages-of-virtualization}
	\section{Cloud computing}
		\subimport{./t/}{t/cloud-introduction}

		\subsection{Service models}

		\subsection{Network in cloud}
			% overlays
		\subsection{Storage in cloud}
			\subimport{./t/}{t/cloud-storage}
			% file storage / block storage

			% share-nothing
			
			% centralized storage
				% NFS mount


			% elasticity

		\subsection{Orchestration software}
			
			\subsubsection{OpenNebula}

	\section{Migration of VMs}

	\section{Distributed data center}

	\pagebreak

	%Sharing resources - maximization of effectiveness
	%Public/private/hybrid
	%metrics
	%SOA - service oriented architecture
	%Routing requests to virtual machines
	%Network environment
	%	overlays
	%Data storage
	%Migration
	%	cold/live
	%Measurement methods
	%Economy
	%	capex -> opex
	%	pay as you goo
\chapter{Practical part}

	Methodology overview

	Framework

	Results

%zdroje
%	\subimport{/}{zdroje}

%seznam zkratek
\printglossary[type=acronym,title=List of Abbreviations,toctitle=List of Abbreviations]

%seznam obrázků
\newpage
\phantomsection \label{listoffig}
\addcontentsline{toc}{chapter}{List of Figures}
\listoffigures

%seznam tabulek
\newpage
\phantomsection \label{listoftab}
\addcontentsline{toc}{chapter}{List of Tables}
\listoftables
\cleardoublepage

\bibliography{citations}{}
\bibliographystyle{plain}

%seznam pojmů
%\printglossary[title=Acronyms,toctitle=Acronyms used]


% přílohy
%\appendix
%\chapter{Příloha}

%\section{Konfigurace prvků}
%\label{priloha_konfigurace}
%{\scriptsize 
%\subsection{R1}
%	\verbatimtabinput{prakticka/conf/R1_P2.conf}
%\subsection{R2}
%	\verbatimtabinput{prakticka/conf/R2_P2.conf}
%\subsection{R3}
%	\verbatimtabinput{prakticka/conf/R3_P2.conf}
%}


%\cleardoublepage
%\section{Úvod do JUNOSu}
%\label{priloha_manual}
%V~následující části je vložen manuál, který popisuje základní ovládání a konfiguraci operačního systému \Zkr{JUNOS}. Slouží k~doplnění praktické části, ve které jsem se soustředil především na samotnou konfiguraci, zatímco základní nastavení zařízení a ovládání je popsáno v~této příloze.
%\includepdf[pages=-]{../branches/junos_manual/manual_import.pdf}

%\cleardoublepage
%\section{Address autoconfiguration in IPv6 networks}
%\includepdf[pages=-]{../branches/clanek/tsp12/clanek_import.pdf}


\end{document}
