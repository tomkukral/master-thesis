%pdflatex-thesis.tex
% vim:spell spelllang=en_us

I have analyzed and compared networking and storage technologies used in distributed (cloud) datacenters. Virtualization is mentioned in the beginning as an enabling technology for cloud computing. Cloud deployment and service models are compared and appropriate use cases are mentioned.

The biggest part is dedicated cloud computing, especially to networking, storage and orchestration technologies. Comparison is made by defining use cases and commenting advantages and disadvantages. All


It is not possible to rigorously decide on best solution because there is so many use cases and every use case require individual approach.

% networking

% load balancing

% storage

% orchestration
Orchestration is described in theoretical part and special attention is given to OpenNebula because it is used in practical part. 


% different requirements not possible 
% pros and cons


% use case for themis

\hrulefill


I have developed application called Themis. The application is capable to evaluate virtual machines availability during live migration. It combines network measurements, orchestration and data analysis.

Migration schema can be defined using console or web interface. Virtual machine availability is then evaluated according to schema. Repetitive migrations are supported as well as configurable bandwidth for packet generator. 


Application configures virtual machine via \Ac{SSH}, starts measurement session, request migration via orchestrator \Ac{API} and collects results. Whole process is fully automated, so no manual configuration is need. 



% tool to measure virtual machines migrations
% network measurement & orchestration & data collection
% sample data in appendix

% Fork

% elmag fork
Fork of the application is used in project Department of Electromagnetic Field. It is used for automatic measurement of wireless links and link monitoring. Migration routines are not used and a remote management is replaced by agent responsible for continuous packet generation. I am going to continue in development of this fork and new features will be merged into main branch.

I am going to continue 

