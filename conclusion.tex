%pdflatex-thesis.tex
% vim:spell spelllang=en_us

I have analyzed and compared networking and storage technologies used in distributed (cloud) datacenters. Virtualization is mentioned in the beginning because it is an enabling technology for the cloud computing. Cloud deployment and service models are compared and appropriate use cases are mentioned.

The biggest attention is dedicated to cloud computing, especially to networking, storage and orchestration technologies. Comparison is made by defining use cases and commenting advantages and disadvantages. It is not possible to rigorously decide on best solution because there are many use cases and each require individual approach.

% networking
Networking is essential part of datacenter design and legacy technologies are not able to fulfill current demand. Overlays techniques are supposed to provide required flexibility and network virtualization. \Ac{VXLAN}, \Ac{STT} and \Ac{NVGRE} is discussed. Hop-by-hop network virtualization is mentioned too, but it is tightly tied to \Ac{SDN} and \Ac{SDN} is still kind of sci-fi technology, because it is not widely supported.  
% load balancing
I have found out that load balancing is important topic connected with overlay networks and \Ac{VM} migrations so I have included load balancing into networking section too.

% storage

% orchestration
Orchestration is described in theoretical part and special attention is given to OpenNebula because it is used in practical part. I have explained principles of orchestration with special focus on virtual networks. Contextualization packages prepared to be used in practical part are presented.

I think that virtualization brings so many improvements to datacenters that legacy networking technologies are not able to keep pace. Most of technologies currently used in network layer were designed before virtualization era and it is the reason why they are to rigid to meet nowadays requirements. It is common to live migrate virtual machine nowadays but it was impossible in a few years before.

We are trying to build highly agile technology, like virtualized and distributed datacenter, on top of the legacy techniques. This approach is obviously not able to work well and there are two solutions possible. 

First radical solution is to totally redesign current network stack and take current requirements into account. However it is very hard to design solution for all use cases with with respect to future usage. It also does not make sense from economical point of view because all network equipment need to be upgraded or replaced. \Ac{SDN} is, in my opinion, typical example of this technology since it brings amazing new features and it is usually can not be used on legacy devices, because hardware changes are necessary.

Second approach is to build new overlay network on top of existing network. This overlay network can provide additional functionality, but also brings some limitations caused by underlaying physical network. It is, for example, not possible to handle priority packets from overlay networks with special care in underlay network, because underlay knows nothing about overlay network. Another problem is \Ac{BUM} traffic because it is usually problematic to handle and multicast need to be implemented in underlay network.
Overlay networks can immediately bring some significant improvements to datacenter networking, but there is a trade-off. I thing that overlays are appropriate temporary solution, but it would recommend to use hop-by-hop with \Ac{SDN} when available.

% practical
It is possible to migrate virtual machine with really minimal outage so \Ac{VM}s can be moved between hypervisors to optimize performance or minimize energy consumption. It is necessary to think about networking aspect of migration, because transfer degradation will definitely appear during migration. I have developed an application called Themis, which is capable to evaluate virtual machine availability during live migration. It combines network measurements, orchestration and data analysis.

It can be used to measure service interruption during migrations. Typical use case is migration of virtual machine with \Ac{SLA}. It is necessary to measure service disruption before migration because there are limits, for example for packet loss, defined in \Ac{SLA}. Themis can be used to migrate testing virtual machine and check whether service degradation during migration is acceptable.

Migration schema can be defined using console or web interface. Virtual machine availability is then evaluated according to schema. Repetitive migrations are supported as well as configurable bandwidth for packet generator. 


Application configures virtual machine via \Ac{SSH}, starts measurement session, request migration via orchestrator \Ac{API} and collects results. Whole process is fully automated, so no manual configuration is need. 



% tool to measure virtual machines migrations
% network measurement & orchestration & data collection
% sample data in appendix

% Fork

% elmag fork
Fork of the application is used in Department of Electromagnetic Field, CTU FEE, for automatic measurement of wireless links and monitoring. Migration routines are not used and remote management is replaced by agent responsible for continuous packet generation. I am going to continue in development of this fork and new features will be merged into main branch.

