%pdflatex-../thesis.tex
% vim:spell spelllang=en_us

It is possible find many services called \Uv{cloud based} and it is important to agree on accurate definition of these services. It is quite clear, that cloud based service will use principle of cloud computing. Definition of cloud computing by \Ac{NIST} says, that \Uv{Cloud computing is a model for enabling ubiquitous, convenient, on-demand network access to a shared pool of configurable computing resources (e.g., networks, servers, storage, applications and services) than can be rapidly provisioned and released with minimal management effort or service provide interaction.} [TODO: cite NIST definition]. This definition clarifies what cloud computing is, but says nothing about parameters and used technologies.

I think, that it would be convenient to start definition from lower levels, which provides elementary parts, and get to the cloud service afterwards. Basic part of cloud computing system is virtual machine (\Ac{VM}). Physical machine can also be part of the cloud system, but it is not able to deliver required rapid provisioning and it is not possible to deploy physical machine without service provider interaction. Virtual machine is elemental resource and also use some additional resources. There resources can be networking, which is used for interconnection between \Ac{VM}s as well as for reaching customers, storage used for system internal or customer data. It is important do employ some configuration management, because it is able to deliver rapid provisioning of virtual machines and minimizes effort required for administration.

Virtual machines together provides the services, which is exposed to users via any kind of networking. Is is common and advisable to provide customer with any kind of management of the service or virtual machines. Difference between cloud computing and bare virtualization is intelligence included in cloud, because it may be controlled automatically according to events or monitoring observed at cloud system. 

Just one \Ac{VM} can be considered cloud based, but it does not allow any scalability. Providing customers with just one \Ac{VM} should not be called cloud computing, but to my way of thinking it is only bare virtualization.

My definition gives different view of cloud computing, but it uses same conditions as \Ac{NIST}'s does and therefore we basically get same result.


According to [TODO: citation of cloud usage], ladslfjadfadfasdfasfs. This means, that almost every service running at the Internet is located in cloud or it will happen in short period. 

First this we should think about is definition of service in cloud. [TODO: definition of server] Therefore cloud service is basically one or more virtual machines running some software, which provides the service. It can be said, that the cloud service is kind of abstraction of virtual machines providing the service and therefore service is equal to some set of virtual machines with suitable software.


